\begin{thebibliography}{10}
\addcontentsline{toc}{chapter}{\bibname}

\bibitem{Atkins}
ATKINS, P. W. \& DE PAULA, J. \textit{Atkins' physical chemistry}. 9th ed. Oxford: Oxford University Press, 2010. ISBN 978-0-19-954337-3.

\bibitem{Leach}
LEACH, A. R. \textit{Molecular modelling: principles and applications}. 2nd ed. New York: Prentice Hall, 2001. ISBN 0-582-38210-6.

\bibitem{Gasteiger:Textbook}
GASTEIGER, J. \& ENGEL, T. \textit{Chemoinformatics: A Textbook}. Weinheim: Wiley-VCH, c2003. ISBN 978-3-527-30681-7.
%„“

\bibitem{molsimul}
RAPPE, A. K. \& GODDARD, W. A. „Charge equilibration for molecular dynamics simulations“. \textit{The Journal of Physical Chemistry}. 1991, \textbf{95}(8), 3358-3363. DOI: 10.1021/j100161a070.

\bibitem{virtscreen}
LYNE, P. D. „Structure-based virtual screening: an overview“. \textit{Drug Discovery Today}. 2002, \textbf{7}(20), 1047-1055. DOI: 10.1016/S1359-6446(02)02483-2.

\bibitem{farmak}
KARELSON, M., LOBANOV, V. S. \& KATRITZKY, A. R. „Quantum-Chemical Descriptors in QSAR/QSPR studies“. \textit{Chemical Reviews}. 1996, \textbf{96}(3), 1027-1044. DOI: 10.1021/cr950202r.

\bibitem{Ghaf:QSAR}
GHAFOURIAN, T. \& DEARDEN, J. C. „The Use of Atomic Charges and Orbital Energies as Hydrogen-bonding-donor Parameters for QSAR Studies: Comparison of MNDO, AM1 and PM3 Methods“. \textit{Journal of Pharmacy and Pharmacology}. 2000, \textbf{52}(6), 603-610. DOI: 10.1211/0022357001774435.

\bibitem{QSPR2}
VAŘEKOVÁ, R. S., GEIDL, S., IONESCU, C.-M., SKŘEHOTA, O., BOUCHAL, T., SEHNAL, D., ABAGYAN, R. \& KOČA, J. „Predicting pK a values from EEM atomic charges“. \textit{Journal of Cheminformatics}. 2013, \textbf{5}(1). DOI: 10.1186/1758-2946-5-18.

%„“
\bibitem{innorg}
WANG, B., LI, S. L. \& TRUHLAR. D. G. „Modeling the Partial Atomic Charges in Inorganometallic Molecules and Solids and Charge Redistribution in Lithium-Ion Cathodes“. \textit{Journal of Chemical Theory and Computation}. 2014, \textbf{10}(12), 5640-5650. DOI: 10.1021/ct500790p.

%%%%%%%%% Kvantove metody %%%%%%%%%%%%%%%%%%%%%%%%%%%%

\bibitem{Cely}
CELÝ, J. \textit{Základy kvantové mechaniky pro chemiky: I. Principy}. Brno: Rektorát UJEP Brno, 1986.

\bibitem{elstat_pot}
HÖLTJE, H.-D. \& FOLKERS, G. \textit{Molecular Modeling: Basic Principles and Applications}. Volume 5. Weinheim: Wiley-VCH, 2008. ISBN 978-3-527-61476-9.

\bibitem{td}
CURTISS, L. A., REDFERN, P. C. \& FRURIP, D. J. „Theoretical Methods for Computing Enthalpies of Formation of Gaseous Compounds“. LIPKOWITZ, K. B. \& BOYD, D. B., ed. \textit{Reviews in Computational Chemistry}. Hoboken, NJ, USA: John Wiley \& Sons, 2000, s. 147-211. \textit{Reviews in Computational Chemistry}. DOI: 10.1002/9780470125922.ch3.

\bibitem{Volatron}
JEAN, Y., VOLATRON, F. \& BURDETT, J. K. \textit{An introduction to molecular orbitals}. New York: Oxford University Press, 1993. ISBN 0-19-506918-8.

\bibitem{BO_approx_Pilar}
PILAR, F. L. \textit{Elementary quantum chemistry}. Dover ed. Mineola, N.Y.: Dover Publications, 2001. ISBN 0-486-41464-7.

\bibitem{qc_complexity}
„Hierarchy of \textit{ab initio} Post-HF metods“. Přetisknuto s laskavým svolením Prof. Martina Kauppa, TU Berlín. \textit{In} SEMRÁD, H. „Studium mechanismu bromoborační reakce metodami kvantové chemie“. Brno: Masarykova univerzita, Přírodovědecká fakulta, Brno, 2016 [online]. [cit. 2019-04-25]. URL: \url{https://is.muni.cz/th/375827/prif_m/}

\bibitem{Cramer}
CRAMER, C. J. \textit{Essentials of Computational Chemistry: Theories and Models}. Chichester: John Wiley, 2002. ISBN 0-471-48552-7.

\bibitem{Levine}
LEVINE, I. N. \textit{Quantum Chemistry}. 6th ed. Upper Saddle River, NJ: Pearson Prentice Hall, 2000. \textcolor{red}{CHYBÍ ISBN}

\bibitem{dft_nmr}
SEFZIK, T. H., TURCO, D., IULIUCCI, R. J. \& FACELLI, J. C. „Modeling NMR Chemical Shift: A Survey of Density Functional Theory Approaches for Calculating Tensor Properties“. \textit{The Journal of Physical Chemistry}. 2005, \textbf{109}(6), 1180-1187. \textcolor{red}{CHYBÍ DOI}

\bibitem{dft}
KOCH, W. \& HOLTHAUSEN, M. C. \textit{A chemist's guide to density functional theory}. New York: Wiley-VCH, 2000. ISBN 3527299181.

\bibitem{CNDO}
POPLE, J. A. \& SEGAL, G. A. „Approximate Self‐Consistent Molecular Orbital Theory. II. Calculations with Complete Neglect of Differential Overlap“. \textit{The Journal of Chemical Physics}. 1965, \textbf{43}(10), 136-151. DOI: 10.1063/1.1701476.

\bibitem{INDO}
POPLE, J. A., BEVERIDGE, D. L. \& DOBOSH, P. A. „Approximate Self‐Consistent Molecular‐Orbital Theory. V. Intermediate Neglect of Differential Overlap“. \textit{The Journal of Chemical Physics}. 1967, \textbf{47}(6), 2026-2033. DOI: 10.1063/1.1712233.

\bibitem{MNDO}
DEWAR, M. J. S. \& THIEL, W. „Ground states of molecules. 38. The MNDO method. Approximations and parameters“. \textit{Journal of the American Chemical Society}. 1977, \textbf{99}(15), 4899-4907. DOI: 10.1021/ja00457a004.

\bibitem{basis_set}
DAVIDSON, E. R. \& FELLER. D. „Basis set selection for molecular calculations“. \textit{Chemical Reviews}. 1986, \textbf{86}(4), 681-696. DOI: 10.1021/cr00074a002. 

\bibitem{MPA}
MULLIKEN, R. S. „Electronic Population Analysis on LCAO–MO Molecular Wave Functions. I“. \textit{The Journal of Chemical Physics}. 1955, \textbf{23}(10), 1833-1840. DOI: 10.1063/1.1740588. 
\bibitem{NPA}
REED, A. E., WEINSTOCK, R. B. \& WEINHOLD, F. „Natural population analysis“. \textit{The Journal of Chemical Physics}. 1985, \textbf{83}(2), 735-746. DOI: 10.1063/1.449486.

\bibitem{AIM}
BADER, R. F. W. \textit{Atoms in molecules: a quantum theory}. New York: Oxford University Press, 1994. ISBN 978-0198558651.

%%%%%%%%%%%%%%% Empiricke metody %%%%%%%%%%%%%%%%%
\bibitem{GM}
GASTEIGER, J. \& MARSILI, M. „Iterative partial equalization of orbital electronegativity—a rapid access to atomic charges“. \textit{Tetrahedron}. 1980, \textbf{36}(22), 3219-3228. DOI: 10.1016/0040-4020(80)80168-2.

\bibitem{MPEOE_aromatic}
PARK, J. M., NO, K. T., JHON, M. S. \& SCHERAGA, H. A. „Determination of net atomic charges using a modified partial equalization of orbital electronegativity method. III. Application to halogenated and aromatic molecules“. \textit{Journal of Computational Chemistry}. 1993, \textbf{14}(12), 1482–1490. DOI:10.1002/jcc.540141210.

\bibitem{GDAC}
CHO, K.-H., KANG, Y. K., NO, K. T. \& SCHERAGA, H. A. „A Fast Method for Calculating Geometry-Dependent Net Atomic Charges for Polypeptides“. \textit{The Journal of Physical Chemistry B}. 2001, \textbf{105}(17), 3624-3634. DOI: 10.1021/jp0023213.

\bibitem{PEOE_nmr}
GASTEIGER, J. \& SURYANARAYANA, I. „A quantitative empirical treatment of 13C NMR chemical shift variations on successive substitution of methane by halogen atoms“. \textit{Magnetic Resonance in Chemistry}. 1985, \textbf{23}(3), 156-157. DOI: 10.1002/mrc.1260230304. 

%%%%%%%% eem 
\bibitem{eem}
MORTIER, W. J., GHOSH, S. K. \& SHANKAR, S. „Electronegativity-equalization method for the calculation of atomic charges in molecules“. \textit{Journal of the American Chemical Society}. 1986, \textbf{108}(15), 4315-4320. DOI: 10.1021/ja00275a013.

\bibitem{sfkeem}
CHAVES, J., BARROSO, J. M., BULTINCK, P. \& CARBÓ-DORCA, R. „Toward an Alternative Hardness Kernel Matrix Structure in the Electronegativity Equalization Method (EEM)“. \textit{Journal of Chemical Information and Modeling}. 2006, \textbf{46}(4), 1657-1665. DOI: 10.1021/ci050505e.

\bibitem{abeem1}
YANG, Z.-Z. \& WANG, C.-S. „Atom−Bond Electronegativity Equalization Method. 1. Calculation of the Charge Distribution in Large Molecules“. \textit{The Journal of Physical Chemistry A}. 1997, \textbf{101}(35), 6315-6321. DOI: 10.1021/jp9711048. 

\bibitem{abeem2}
WANG, C.-S. \& YANG, Z. Z. „Atom–Bond Electronegativity Equalization Method. II. Lone-pair electron model“. \textit{The Journal of Chemical Physics}. 1999, \textbf{110}(13), 6189-6197. DOI: 10.1063/1.478524.

\bibitem{zeolites}
HEIDLER, R., JANSSENS, G. O. A., MORTIER, W. J. \& SCHOONHEYDT, R. A. „Charge Sensitivity Analysis of Intrinsic Basicity of Faujasite-Type Zeolites Using the Electronegativity Equalization Method (EEM)“. \textit{The Journal of Physical Chemistry}. 1996, 100(50), 19728-19734. DOI: 10.1021/jp9615619.

\bibitem{eem_protein}
IONESCU, C.-M., GEIDL, S., VAŘEKOVÁ, R. S. \& KOČA, J. „Rapid Calculation of Accurate Atomic Charges for Proteins via the Electronegativity Equalization Method“. \textit{Journal of Chemical Information and Modeling}. 2013, \textbf{53}(10), 2548-2558. DOI: 10.1021/ci400448n.

\bibitem{guided_m}
PAZÚRIKOVÁ, J., KŘENEK, A. \& MATYSKA, L. „Guided Optimization Method for Fast and Accurate Atomic Charges Computation“. In ÉVORA-GÓMEZ, J. \& HERNANDÉZ-CABRERA, J. J. \textit{Proceedings of the 2016 European Simulation and Modelling Conference}. Ghent: EUROSIS - ETI, 2016. 267-274. ISBN 978-90-77381-95-3

%%%%%%%%%% atomove typy
\bibitem{attyp_peptides}
KANG, Y. K. \& SCHERAGA, H. A. „An Efficient Method for Calculating Atomic Charges of Peptides and Proteins from Electronic Populations“. \textit{The Journal of Physical Chemistry B}. 2008, \textbf{112}(17), 5470-5478. DOI: 10.1021/jp711484f.

\bibitem{attyp2}
YAKOVENKO, O., OLIFERENKO, A. A., BDZHOLA, V. G., PALYULIN, V. A. \& ZEFIROV, N. S. „Kirchhoff atomic charges fitted to multipole moments: Implementation for a virtual screening system“. \textit{Journal of Computational Chemistry}. 2008, \textbf{29}(8), 1332-1343. DOI: 10.1002/jcc.20892.

%%%%%%%%%%%%%%% Statisticke pojmy
\bibitem{oxford}
DODGE, Y. \textit{The Oxford Dictionary of Statistical Terms}. Oxford: Oxford University Press, 2003. ISBN 978-0199206131.

\bibitem{rmsd}
MAIOROV, V. N. \& CRIPPEN, G. M. „Significance of Root-Mean-Square Deviation in Comparing Three-dimensional Structures of Globular Proteins“. \textit{Journal of Molecular Biology}. 1994, \textbf{235}(2), 625-634. DOI: 10.1006/jmbi.1994.1017. 

\bibitem{PCC}
PAVLÍK, T. \& DUŠEK, L. \textit{Biostatistika}. Brno: Akademické nakladatelství CERM, 2012. ISBN 978-80-7204-782-6.

%%%%%%%%%%%%%%% METODY %%%%%%%%%%%%%%%%%%
\bibitem{sdf_pdf}
\textit{CTFile Formats} [online]. [cit. 2019-04-18]. URL: \url{http://c4.cabrillo.edu/404/ctfile.pdf}

\bibitem{sdf_clanek}
DALBY, A., NOURSE, J. G., HOUNSHELL, W. D., GUSHURST, A. K. I., GRIER, D. L., LELAND, B. A. \& LAUFER, J. „Description of several chemical structure file formats used by computer programs developed at Molecular Design Limited“. \textit{Journal of Chemical Information and Modeling}. 1992, \textbf{32}(3), 244-255. DOI: 10.1021/ci00007a012.

% PDB format
\bibitem{PDB1}
Worldwide Protein Data Bank: \textit{Protein Data Bank Contents Guide: Atomic Coordinate Entry Format Description, Version 3.30} [online]. [cit. 2019-04-30]. URL: \url{ftp://ftp.wwpdb.org/pub/pdb/doc/format_descriptions/Format_v33_Letter.pdf}

\bibitem{PDB2}
UCSF: Resource for Biocomputing, Visualization, and Informatics: \textit{Introduction to Protein Data Bank Format} [online]. [cit. 2019-04-30]. URL: \url{https://www.cgl.ucsf.edu/chimera/docs/UsersGuide/tutorials/pdbintro.html}

% nomenklatura atomovych typu aminokyselin
\bibitem{AA_nomenclature}
MARKLEY, J. L., BAX, A., ARATA, Y., HILBERS, C. W., KAPTEIN, R., SYKES, B. D., WRIGHT, P. E. \& WUTHRICH, K. „Recommendations for the presentation of NMR structures of proteins and nucleic acids. IUPAC-IUBMB-IUPAB inter-union task group on the standardization of data bases of protein and nucleic acid structures determined by NMR spectroscopy“. \textit{European Journal of Biochemistry}. 1998, \textbf{256}(1), 1-15. DOI: 10.1046/j.1432-1327.1998.2560001.x.

% SMILES %
\bibitem{Weininger}
WEININGER, D. „SMILES, a chemical language and information system. 1. Introduction to methodology and encoding rules“. \textit{Journal of Chemical Information and Modeling}. 1988, \textbf{28}(1), 31-36. DOI: 10.1021/ci00057a005.

\bibitem{Bunin}
BUNIN, B. A. \textit{Chemoinformatics: Theory, Practice, \& Products}. Dordrecht: Springer, 2007. ISBN 987-1-4020-5000-8.

\bibitem{Leach_chemo}
LEACH, A. R. \textit{An Introduction to Chemoinformatics}. Dordrecht: Springer, 2007. ISBN 978-1-4020-6290-2.

\bibitem{SMILES_exm}
DAYLIGHT: \textit{SMILES - A Simplified Chemical Language} [online]. 2008 [cit. 2019-04-20]. URL: \url{ http://www.daylight.com/dayhtml/doc/theory/theory.smiles.html}

% SMARTS %
\bibitem{SMARTS_intro}
DAYLIGHT: \textit{SMARTS - A Language for Describing Molecular Patterns} [online]. 2008 [cit. 2019-04-20]. URL: \url{ http://www.daylight.com/dayhtml/doc/theory/theory.smarts.html}

\bibitem{SMARTS_exm}
DAYLIGHT: \textit{SMARTS Examples} [online]. 2008 [cit. 2019-04-20]. URL: \url{ http://www.daylight.com/dayhtml_tutorials/languages/smarts/smarts_examples.html#X}
% RDKit
%\bibitem{rdk_org}
%\textit{RDKit: Open-Source Cheminformatics Software} [online]. 2018 [cit. 2019-04-20]. URL: \url{https://www.rdkit.org/}

\bibitem{rdk_info}
RDKit: \textit{An overview of the RDKit} [online]. 2018 [cit. 2019-04-20]. URL: \url{ https://www.rdkit.org/docs/Overview.html}

\bibitem{Postgre}
PostgreSQL: The World's Most Advanced Open Source Relational Database [online]. [cit 2019-05-02]. URL: \url{https://www.postgresql.org/}

% MACH %
\bibitem{mach}
RAČEK, T., SCHINDLER, O., SVOBODOVÁ VAŘEKOVÁ, R. \& KOČA, J. „Empirical methods for calculation of partial atomic charges - applicability for proteins?“. In \textit{ENBIK2018}. 2018. ISBN 978-80-7592-017-1.

\bibitem{scipy}
SciPy.org [online]. [cit 2019-05-02]. URL: \url{https://www.scipy.org/}

\bibitem{nlopt}
NLopt Documentation [online]. [cit 2019-05-02].
URL: \url{http://ab-initio.mit.edu/nlopt}

\bibitem{acsk2}
VERSTRAELEN, T., AYERS, P. W., VAN SPEYBROECK, V. \& WAROQUIER, M. „ACKS2: Atom-condensed Kohn-Sham DFT approximated to second order“. \textit{The Journal of Chemical Physics} 2013, \textbf{138}(7). DOI: 10.1063/1.4791569.

\bibitem{mgc}
OLIFERENKO, A. A., PALYULIN, V. A., PISAREV, S. A., NEIMAN, A. V. \& ZEFIROV, N. S. „Novel point charge models: reliable instruments for molecular electrostatic“. \textit{Journal of Physical Organic Chemistry}. 2001, \textbf{14}(6), 355-369. DOI: 10.1002/poc.378.

%%%%%%%% IMPLEMENTACE %%%%%%%%%
% substructures
\bibitem{attyp1}
GILSON, M. K, GILSON, H. S. R. \& POTTER, M. J. „Fast Assignment of Accurate Partial Atomic Charges: An Electronegativity Equalization Method that Accounts for Alternate Resonance Forms“. \textit{Journal of Chemical Information and Computer Sciences}. 2003, \textbf{43}(6), 1982-1997. DOI: 10.1021/ci034148o.

\end{thebibliography}


\cleardoublepage
