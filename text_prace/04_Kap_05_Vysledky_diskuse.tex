\chapter{Výsledky a diskuse}
Kapitola shrnuje a interpretuje výsledky parametrizací metod EEM a PEOE za užití navržených klasifikací atomů. 
%Získané výsledky jsou průběžně interpretovány. 
Veškeré výsledky parametrizací lze nalézt v externí příloze bakalářské práce nebo na adrese \url{https://lcc.ncbr.muni.cz/~raduse19/}.

\section{Vstupní data}
Pro parametrizaci empirických metod byly použita molekulová sada ve formátu SDF obsahující 4 443 molekul o 204 760 atomech a sada 32 molekul ve formátu PDB čítající 29 107 atomů. Pro analýzu přenositelnosti parametrů byly molekuly v molekulových sadách rozděleny na sadu tréninkovou a validační v poměru 9:1, přičemž obě sady obsahovaly identické atomové typy. Klasifikace určené pro formát SDF byly nejdříve zkušebně aplikovány na tréninkovou sadu obsahující 500 molekul a použity pro parametrizaci. Empirické metody byly pro obě molekulové sady parametrizovány vůči referenčním nábojům B3LYP/6-311Gd/NPA.

\section{Výsledky parametrizace}
Pro každý implementovaný klasifikátor je výsledek dané parametrizace srovnán s parametrizací využívající klasifikaci atomů dle protonového čísla. 


\subsection{Klasifikace atomů formátu SDF}
- tabulka o 2 sloupcích pro EEM a PEOE 

- do přílohy dát 
\subsection{Klasifikace atomů formátu PDB}
\subsection{Úprava klasifikátorů 'substruct' a 'peptide'}
\section{Srovnání výsledků parametrizací empirických metod}