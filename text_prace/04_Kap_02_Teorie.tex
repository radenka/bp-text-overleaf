\chapter{Teorie}
% Druhá kapitola s matematikou \texorpdfstring{$\int\! f(x)\,\mathrm{d}x$}{int f(x) dx} v~názvu
%% matematicke symboly nelze do zalozek v PDF vlozit - proto slouzi prikaz \texorpdfstring{toto se vysazi}{toto se
%% vlozi do zalozek v PDF} - nektere symboly vlozit lze, viz kapitolu 50 v dokumentaci
%% http://mirrors.ctan.org/macros/latex/contrib/hyperref/hyperref.pdf
%% viz take http://orgmode.org/worg/org-symbols.html
%%

%%%%%%%%%%%%%%%%%%%%%%%%%%%%%%%%%%%%%%%%%%%%%%%%%%%%%%%%%%%%%%
%%%%%%%%% UKAZKA OPAKOVANI MATEMATICKYCH SYMBOLU %%%%%%%%%%%%%
% 
% ukázka opakování na řádkovém zlomu ukázka opakování na řádkovém zlomu zlom $c+ a+ b$
% 
% ukázka opakování na řádkovém zlomu ukázka opakování na řádkovém zlomu zl $c- a- b$
% 
% ukázka opakování na řádkovém zlomu ukázka opakování na řádkovém zlomu zlo $c\cdot a\cdot b$
% 
% ukázka opakování na řádkovém zlomu ukázka opakování na řádkovém zlomu zl $c\setminus a\setminus b$
% 
Teoretická část práce seznamuje čtenáře s konceptem parciálních atomových nábojů a~s metodami jejich výpočtu. Blíže popisuje teoretické základy empirických metod EEM a PEOE, které byly použity v praktické části práce, podobně jako uvedené statistické veličiny.

\section{Parciální atomové náboje}

Parciální atomové náboje jsou reálná čísla, která popisují asymetrické rozložení elek\-tronové hustoty na chemické vazbě \cite{Atkins}. Vznikají v důsledku rozdílných elektronegativit vazebných partnerů. Pokud v chemické vazbě figuruje vysoce elektronegativní atom, pak tento k sobě přitahuje vazebný elektronový pár, čímž se zvyšuje elektronová hustota v~ jeho okolí a dochází ke vzniku parciálního záporného náboje ($\delta$-). V okolí elektropozitivnějšího vazebného partnera se elektronová hustota naopak snižuje a~na atomu dochází ke vzniku parciálního kladného náboje ($\delta$+). 

\bigskip
\begin{figure}[h]
\begin{center}
\includegraphics[width=6cm]{example-image-a}
\caption{Rozložení elektronové hustoty v molekule ethanolu.}
\end{center}
\end{figure}
Koncept parciálních atomových nábojů je pouze teoretický, hodnoty nábojů proto nelze získat pomocí experimentu \cite{Leach}. Jelikož se parciální atomové náboje uplatňují při predikci fyzikálních, chemických a biologických vlastností molekul, bylo pro jejich stanovení vyvinuto množství výpočetních metod. Tyto se dělí na metody kvantově-chemi\-cké a metody empirické. Kvantově-chemické metody představují standardní přístup výpočtu parciálních atomových nábojů,  avšak použití těchto metod může být limitováno jejich velkou časovou náročností. Empirické metody pro výpočet parciálních atomových nábojů používají v rámci výpočtů parametrizovaná data a představují tak časově méně náročnou alternativu kvantově-chemického přístupu \cite{Gasteiger:Textbook}. 
% Kvantově-chemické metody poskytují přesnější výsledky, ovšem za cenu vysoké časové náročnosti. Empirické metody dosahují v porovnání s QM metodami velmi dobrých výsledků, a to ve výrazně kratším čase. 
%Žádná z vyvinutých empirických metod však není uznána za všeobecně platnou a jejich použitelnost se hodnotí na základě reprodukovatelnosti výsledků. 

Aplikaci parciálních atomových nábojů lze nalézt ve výpočetní chemii a chemoinformatice, kde slouží k predikci elektrostatických vlastností popisujících reaktivitu molekul. Uplatňují se v molekulových simulacích \cite{molsimul}, ve virtuálním screeningu \cite{virtscreen}, při hledání vazebných míst proteinů nebo při návrhu farmakoforů \cite{farmak}. Prokázaly se jako platné deskriptory v QSAR a QSPR modelech \cite{Ghaf:QSAR, QSPR2}. V anorganické chemii se uplatňují při popisu toku elektronů v bateriích a katalyzátorech \cite{innorg}. 

%%%%%%%%%%%%%%% QC METODY %%%%%%%%%%%%%%%%%%%%%%%%%%%%%
\section{Kvantově-chemické metody}
V této podkapitole je popsán teoretický aparát kvantové mechaniky, na který dále navazuje popis kvantově-chemických metod pro výpočet parciálních atomových nábojů. 

\subsection{Základy kvantové mechaniky}
%Již na konci 19. století došli vědci k poznání, že klasická newtonovská mechanika není vhodná pro popis pohybu mikročástic (např. elektronů). Z toho důvodu došlo ve 20. letech 20. století k rozvoji kvantové mechaniky. 
Kvantová mechanika se rozvinula ve 20. letech 20. století v reakci na newtonovskou mechaniku, jejíž aparát již nepostačoval pro popis mikrosvěta.  Základním principem QM je vlnově-korpuskulární dualismus, který mikročástici připisuje jak charakteristiky hmoty (hybnost), tak charakteristiky elektromagnetické vlny šířící se prostorem.
Vlna, popisující částici, je v kvantové mechanice reprezentována matematickou funkcí $\Psi$, tzv. vlnovou funkcí. Tato funkce popisuje dynamický stav částice a nese informaci o výskytu částice v prostoru. Vlnová funkce elektronu tak přispívá k popisu rozložení elektronů v molekule \cite{Cely}. 

Základním úkolem kvantové mechaniky je výpočet vlnové funkce systému, z níž lze odvodit elektrostatický potenciál \cite{elstat_pot} nebo termodynamické vlastnosti molekuly \cite{td}. Vlnová funkce je řešením Schrödingerovy rovnice
\begin{equation}
    H\Psi = E\Psi
\end{equation}
kde $H$ je operátor hamiltonián a $E$ je energie systému. Hamiltonián působí vlnovou funkci $\Psi$ a transformuje ji na funkci jinou. Řešením Schrödingerovy rovnice je soubor funkcí, které lze po aplikaci hamiltoniánu zapsat jako součin původní funkce a~skaláru $E$. Takovéto funkce označujeme jako \textit{vlastní funkce} a odpovídající skaláry jako \textit{vlastní hodnoty} operátoru \cite{Volatron}. 
 %Řešení Schrödingerovy rovnice pro daný systém tedy odpovídá hledání dvojic vlastních funkcí a vlastních hodnot operátoru $H$ vyhovujících rovnici xx.
 
 Schrödingerova rovnice je exaktně řešitelná pouze pro vybrané problémy, např. pro atom vodíku. Pro víceelektronové systémy je nutno do výpočtu zavádět velké množství aproximací, z nichž nejznámější je Born-Oppenheimerova aproximace \cite{BO_approx_Pilar}. % Elementary quantum chemistry, Frank L. Pilar
 Jejím základním konceptem je oddělení řešení SE pro jádra od řešení rovnice elektronů. Tento postup vychází z předpokladu, že jádra atomů, mnohonásobně těžší než elektrony,
 %složená z protonů a neutronů, které jsou oproti elektronům více než 1800krát těžší, 
 se pohybují výrazně pomaleji než elektrony samotné, a lze je tedy pro řešení SE elektronů pokládat za fixní. %Řešení Schrödingerovy rovnice se tak rozkládá na řešení popisující elektrony v souboru fixních jader, po němž následuje řešení rovnice zahrnující kinetickou a potenciální energii jader obklopených polem elektronů.
 
\subsection{nevím kam to dát}
Kvantově-mechanické metody pro výpočet parciálních atomových nábojů jsou založeny na poznatcích kvantové mechaniky. Dělí se na tři hlavní skupiny, a to metody semi-empirické, metody odvozené od teorie funkcionálu hustoty a metody \textit{ab initio}. \textit{Ab initio} metody (lat. \textit{ab initio} - od počátku) staví výpočty na teoretickém aparátu a k řešení Schrödingerovy rovnice přistupují numericky%využívají minimum aproximací
, z čehož vyplývá jejich velká výpočetní náročnost. Metody semi-empirické jsou stejně jako metody \textit{ab initio} založeny na řešení SE, pro zjednodušení výpočtů ale využívají kromě značné míry aproximací také data z experimentu. 
%část výpočtů však parametrizují nebo aproximují na základě experimentálních dat.

Limitujícím faktorem pro použití QM metod je jejich složitost, konkrétně pro \textit{ab initio} metody až $O(B^4)$, kde B je číslo rovno počtu elektronů v molekule nebo větší. 
% !!! U ab-inito metod (sprazene klastry, perturbation theory) slozitost az n^7!! Viz Hugova DP, str 25 https://is.muni.cz/auth/th/otfyl/DP_Semrad_H.pdf
 
 %%%%%%%%%%%%%%%%% PREHLED QC METOD %%%%%%%%%%%%%%%%%%
\subsection{Přehled kvantově-mechanických metod}
Důležitým krokem \textit{ab initio} a semi-empirických metod je výběr bázové sady. Bázová sada je soubor vlnových funkcí reprezentující atomové orbitaly, jejichž vhodnou lineární kombinací (LCAO) lze následně vyjádřit vlnovou funkci molekuly. %Bázová sada obsahující nejmenší možný počet atomových orbitalů potřebných pro konstrukci molekulových orbitalů se označuje jako \textit{minimální báze}. 
Pro popis funkcí reprezentujících atomové orbitaly se používají orbitaly Gaussova typu (GTO). Kombinace několika Gaussových orbitalů přibližuje tzv. Slaterův orbital (STO), který je pro výpočet vlnové funkce molekuly méně vhodný z důvodu složitosti výpočtů. Bázových sad existuje nepřeberné množství, např. bázové sady STO-3G, STO-4G či obecně STO-\textit{n}G, kde \textit{n} je počet orbitalů Gaussova typu reprezentujících jeden atomový orbital.
%sady 6-31G* či 6-311G.

Krokem vedoucím k výpočtu hodnot parciálních nábojů je provedení populační analýzy, která popisuje rozložení elektronové hustoty v molekule. Příkladem je Mullikenova populační analýza (\textit{Mulliken Population Analysis}, MPA), % R. S. Mulliken, “Electronic population analysis on LCAO–MO molecularwave functions. I,”The Journal of Chemical Physics, vol. 23, no. 10, pp. 1833–1840, 1955.
která elektronovou hustotu určuje dle obsazenosti atomových orbitalů  elektrony. V rámci chemické vazby je elektronová hustota rovnoměrně rozdělena mezi vazebné partnery, není tedy brána v potaz možná rozdílnost elektronegativit. Výsledky MPA jsou také silně závislé na použitém kvantově-mechanickém přístupu a na velikosti bázové sady.
Nevýhody MPA, zejména nepřesnost výsledků související s rozšiřováním bázové sady, řeší přirozená populační analýza (\textit{Natural Population Analysis}, NPA), pracující s přirozenými atomovými orbitaly. 
Přirozené atomové orbitaly jsou nejprve vypočteny z bázové sady a jsou následně použity pro výpočet ortonormálních přirozených vazebných orbitalů (\textit{Natural bonding orbitals}, NBO). Na základě NBO se poté provadí populační analýza. %  A. E. Reed, R. B. Weinstock and F. Weinhold. Natural population analysis.J. Chem.Phys. 83(2):735-746, 198

Odlišný přístup finálního výpočtu parciálních atomových nábojů představuje metoda \textit{Atoms-in-Molecules} (AIM), která přiřazuje náboje atomům na základě integrace elektronové hustoty přes prostor příslušící danému atomu. Dalším možným přístupem je výpočet parciálních atomových nábojů na základě elektrostatických potenciálů molekuly (metody založené na \textit{Molecular Electrostatic Potential-derived charges}, MEP). 


\subsubsection{Hartree-Fockova metoda}
Problém řešení víceelektronových systémů nastává při zahrnutí elektronových interakcí do výpočtu. Hartree-Fockova metoda rozkládá původní problém \textit{n}-elektronové Schrödingerovy rovnice na řešení \textit{n} jednoelektronových rovnic. Využívá přiblížení pomocí metody nezávislých částic (\textit{Self-Consistent Field}, SCF), která pracuje s modelem elektronu pohybujícím se v průměrném poli ostatních elektronů. HF metoda tedy nezahrnuje korelaci pohybu elektronů.
% ver2 
%Interakce elektronu s ostatními elektrony je tedy aproximována na působení vnějšího elektronového pole na danou částici.
% ver1 
%Jednotlivé jednoelektronové rovnice tedy zahrnují působení vnějšího elektronového pole. 
Jednoelektronové rovnice jsou určeny předpisem
\begin{equation}
    \hat{F} \chi_i = \varepsilon_i \chi_i
\end{equation}
kde Fockův operátor $\hat{F}$ je Hamiltonián aplikovaný na jednoelektronový (atomový nebo molekulový) orbital a $\varepsilon_i$ je odpovídající Langrangeův multiplikátor. Metoda pracuje iterativně a konečné řešení rovnic určuje na základě ustálení výsledků jednotlivých iterací výpočtu.

% PŮVODNÍ TEXT K SCF: Problém řešení víceelektronových systémů nastává při zahrnutí elektronových interakcí do výpočtu. Interakce elektronů molekuly lze řešit pomocí přiblížení metody nezávislých částic (SCF, Self-Consistent Field), která pracuje s modelem elektronu pohybujícím se v průměrném poli ostatních elektronů. Původní problém se tak rozkládá na řešení jednoelektronových rovnic, zahrnující působení vnějšího elektronového pole. Teorii SCF využívá Hartree-Fockova metoda, označována jako HF-SCF. 

% Jako zdroj uvest Quantum Chemistry, Ira N. Levine, Chapter 11 Many-electron atom;, Computattonal Chemistry, Errol Lewards

\subsubsection{DFT}
Metody založené na teorii funkcionálu hustoty (\textit{Density Functional Theory}, DFT) nevycházejí z řešení vlnové funkce, ale  poznatky staví na rozložení elektronové hustoty v molekule, ze které následně odvozují energii systému a další vlastnosti molekuly. Do výpočtů DFT metod je narozdíl od Hartee-Fockovy metody zahrnuta i korelační energie elektronů. Ve výpočtech figurují pouze tři neznámé (souřadnice \textit{x}, \textit{y}, \textit{z}), zatímco řešení Schrödingerovy rovnice obsahuje 4\textit{n} neznámých, kde \textit{n} představuje počet elektronů systému. Metody odvozené od DFT jsou tak výpočetně méně náročné a poskytují přesnější výsledky.

Jedním z cílů DFT metod je výpočet celkové energie elektronů na základě elektronové hustoty.\textit{ Funkcionál} je v rámci DFT metod chápán jako zobrazení, které zobrazuje funkci, představující elektronovu hustotu, do množiny reálných čísel popisujících energii elektronů. Říkáme, že energie elektronů je funkcionálem elektronové hustoty. %slozitost elektronu SE: Lewards; Gasteiger; Leach; Levine; citovat články od Peti a Bichiho
% Myšlenka DFT: všecky elektronové vlastnosti se získají z elektronové hustoty (např. energie systému/energie základního stavu - Leach) (Peťa BP). Jednodušší v porovnání s výpočtem vlnové funkce, kde je zahrnut ještě popis elektronu. Leach: Cílem DFT je výpočet elektronové hustoty systému a zní získání celkové energie elektronů (electronic energy).
% Typy používaných funkcionálů: Local Density Aproximation (LDA), Generalized Gradient Aproximation (GGA) a Hybrid Functionals (hybridní funkcionály).
\subsubsection{Semiempirické metody}
QM metody byly v době svého vzniku limitovány nedostatečnými výpočetními zdroji. Problém vyřešil rozvoj semi-empirických metod, které část výpočtů parametrizují nebo aproximují na základě experimentálních dat, přičemž se snaží přiblížit QM výpočtům. 

Raná semi-empirická metoda CNDO (\textit{Complete Neglect of Differential Overlap}) využívá teorii SCF pro popis elektronových interakcí a je založena na ZDO aproximaci (\textit{Zero Differential Overlap}). Ta zamítá interakce atomových orbitalů lokalizovaných na různých atomech molekuly a pracuje pouze s interakcemi atomových orbitalů stejného typu lokalizovaných na stejném atomu. 
%Ta dává všechny překryvové integrály popisující čtyři atomové orbitaly lokalizované na různých atomech rovny nule, čímž vyjadřuje, že interakcí těchto orbitalů nevzniká molekulový orbital. 
Tyto hrubé aproximace nahradila metoda INDO  (\textit{Intermediate Neglect of Differential Overlap}) zahrnutím interakcí odlišných typů atomových orbitalů. Dalšími semi-empirickými metodami jsou např. NDDO, MNDO, PM3 nebo SAM1, založené na MNDO. 

%část výpočtů však parametrizují nebo aproximují na základě experimentálních dat.
% Atkins
% integral set to zero = překryvem atomových orbitalů nevzniká molekulový orbital (Atkins český, s. 403)
%- QM ve své době limitovány nedostatečnými výpočetními zdroji, aplikace pouze pro malé molekuly -> SE metody: ve větší míře využití aproximací nebo části výpočtů parametrizují na základě exerimentálních dat, snaží se přiblížit své výsedky QM výpočtům
%- příklady metod: EHT rozšířená Hückelova metoda
%- modernější metody: CNDO (využití SCF teorie pro popis elektronových interakcí), založená na ZDO aproximaci (Zero Differential Overlap) - nejsou brány v potaz překryvy orbitalů odlišného typu, jejich překryv roven nule
%- Atk: CNDO - všechny překryvové integrály popisující čtyři atomové orbitaly lokalizované na různých atomech jsou rovny nule = překryvem daných AO nevzniká MO
%- INDO nahradila CNDO - nahrazení hrubých ZDO aproximací, zahrnutí vzájemných interakcí elektronů přislušících stejnému atomu
%- Další vyvinuté metody: NDDO a z ní odvozená MNDO; dále AM1 nebo PM3
\section{Empirické metody}
Empirické metody výpočtu parciálních atomových nábojů byly vyvinuty v reakci na velkou výpočetní náročnost QM metod. V porovnání s QM metodami dosahují velmi přesných výsledků, a to ve výrazně kratším čase. Empirické metody se dělí na dvě hlavní skupiny, a to metody pracující s topologií molekuly (jinak řečeno s její 2D strukturou) a metody pracující s prostorovým uspořádáním molekuly. Metody zastupující obě uvedené skupiny, jmenovitě metoda PEOE a metoda EEM, jsou popsány v odstavcích níže.
\subsection{PEOE}
Metoda PEOE (\textit{Partial Equalization of Orbital Electronegativity}), známá také pod jménem autorů jako metoda Gasteiger-Marsili, byla poprvé publikována v roce 1980 [c]. Metodu lze aplikovat pouze na systémy obsahující $\sigma$ vazby a nekonjugované $\pi$  vazby, později však byla autory rozšířena i o výpočet systémů konjugovaných $\pi$ vazeb. V rámci metody není uvažována 3D struktura molekuly, pracuje se pouze s její topologií. 

Koncept elektronegativity atomových orbitalů, na němž je metoda založena, vychází z Mullikenovy definice elektronegativity $\chi_A$ atomu \textit{A}
\begin{equation}
    \chi_A = \frac{1}{2}(I_A + E_A)
\end{equation}
Dle Mullikena je elektronegativita atomu určena hodnotami elektronových afinit $ E_A$ a ionizačních potenciálů $I_A$ jeho valenčních stavů. PEOE připisuje na základě hodnot $I_A$ a $ E_A$ elektronegativitu každému orbitalu valenčního stavu atomu. Elektronegativita $\chi_{iv}$ orbitalu \textit{iv} na atomu \textit{i}
\begin{equation}
\label{PEOE_elneg}
    \chi_{iv} = a_{iv} + b_{iv}Q_i + c_{iv}Q_i^2
\end{equation}
je ovlivněna náboji ostatních orbitalů a tedy i celkovým nábojem příslušného atomu $Q_i$. Koeficienty $a_{iv}$, $b_{iv}$ a $c_{iv}$ jsou empirické parametry vypočtené z ionizačních potenciálů a elektronových afinit neutrálního, kationtového a aniontového stavu příslušného orbitalu. (za PEOE vložit tabulku parametrů abc, viz Gasteiger s. 330)

Při vzniku vazby dochází vlivem elektronegativity atomů k přesunu elektronů od elektropozitivnějšího atomu směrem k elektronegativnějšímu. Interagují spolu přísluš\-né atomové orbitaly a dochází k částečné ekvalizaci (vyrovnání) jejich nábojů. Množ\-ství přeneseného náboje mezi atomy \textit{A} a \textit{B} v \textit{k}-té iteraci výpočtu, kde atom \textit{B} má vyšší elektronegativitu, je definováno jako 
\begin{equation}
\label{transfer}
    Q^{\langle k \rangle} = \frac{\chi_B^{\langle k \rangle} - \chi_A^{\langle k \rangle}}{\chi_A^+} \cdot \alpha^k
    % \Bigg(\frac{1}{2}\Bigg)
\end{equation}
kde $\chi_A^+$ označuje elektronegativitu kationtu atomu \textit{A}. Iniciální výpočet elektronegativity orbitalu (\ref{PEOE_elneg}) pracuje s formálním nábojem atomu. Po výpočtu příspěvků přenesených nábojů (\ref{transfer}) všech vazebných partnerů atomu je náboj daného atomu přepočítán a použit v další iteraci. Množství přeneseného náboje mezi dvěma atomy se v každé iteraci výpočtu snižuje a vypočtené hodnoty nábojů atomů postupně konvergují. Přibližně po šesté iteraci dochází k ustálení výpočtů. 

Díky přesnosti a rychlosti výpočtů byla PEOE implementována do většiny programů pro molekulové modelování jako základní metoda výpočtu atomových nábojů. Reziduální elektronegativita atomů, získaná na základě PEOE, se prokázala vhodnou pro popis indukčního efektu v molekulách. 
%náboje vypočtené pomocí PEOE použity ve výpočtech dipolových momentů nebo posunů v rámci 13C NMR; ZDROJE citované v článku PEOE, v Gasteigerovi, Krabík a Bichi DP, disert.

\subsection{EEM}
Autoři Mortier, Ghosh a Shankar publikovali metodu elektronové ekvalizace (\textit{Electronegativity Equalization Method}, EEM) v r. 1986. Teoretický základ metody vychází z teorie funkcionálu hustoty, na němž je vystavěn matematický aparát pro výpočet atomových nábojů. EEM využívá referenční náboje získané kvantově-mechanickými metodami, pomocí nichž parametrizuje vlastní výpočty. Díky nízké výpočetní náročnosti ($\Theta(N^3)$, kde \textit{N} je počet atomů systému) a poměrně přesným výsledkům se stala metoda hojně využívanou a byla použita např. pro výpočet parciálních nábojů zeolitů, organických molekul nebo polypeptidů.

Výchozím konceptem metody je Sandersonův princip ekvalizace elektronegativity. Dle něj je každému atomu přiřazena stejná elektronegativita jako je elektronegativita ostatních atomů molekuly. Podle rovnice 
\begin{equation}
\label{Sanders}
    \overline{\chi} = \chi_1 = \chi_2 = \chi_3 = ... = \chi_N
\end{equation}
kde \textit{N} je počet atomů, se elektronegativita každého atomu rovná průměrné elektronegativitě molekuly $\overline{\chi}$. Sandersonův postulát je potvrzen principy DFT. % možná radši vymazat

Další základním principem metody je princip zachování náboje. Celkový náboj molekuly (\textit{Q}) odpovídá součtu dílčích atomových nábojů $q_i$.
\begin{equation}
\label{EEM_chargesum}
    \sum_{i} q_i = Q
\end{equation}

Třetí základní princip představuje efektivní elektronegativita $\chi_i$ atomu \textit{i}. Jelikož metoda pracuje s prostorovým uspořádáním molekuly, započítává se při určování elektronegativity atomu \textit{i} jeho molekulové okolí. Sumace ve vzorci reprezentuje elektrostatickou interakci atomu \textit{i} s dalšími \textit{N} atomy \textit{j} molekuly  v závislosti na jejich vzdálenosti $R_{ij}$.
\begin{equation}
\label{EEM_elneg}
    \chi_i = A_i + B_i\cdot q_i + \kappa \sum_{j \neq i}^{N} \frac{q_j}{R_{ij}}
\end{equation}
V rovnici kromě nábojů $q_i$, $q_j$ interagujících atomů vystupují empirické parametry $A_i$, $B_i$ a $\kappa$. Parametry  $A_i$ a $B_i$ zahrnují elektronegativitu $\chi_{i}^{0}$ a tvrdost $\eta_{i}^{0}$ neutrálního izolovaného atomu a korekce $\Delta \chi_i$, $\Delta \eta_i$, které upravují výslednou elektronegativitu $\chi_i$ atomu na základě jeho interakce s prostředím molekuly. 
\begin{equation}
    A_i = \chi_{i}^{0} + \Delta \chi_i
\end{equation}
\begin{equation}
        B_i = 2(\eta_{i}^{0} + \Delta \eta_i)
\end{equation}
Cílem metody je empiricky nalézt hodnoty korekcí pro definované atomové typy a zajistit tak znovupoužitelnost uvedených parametrů. 

Řešení systému s \textit{N} atomy vede po kombinaci vztahů  \ref{Sanders}, \ref{EEM_chargesum} a \ref{EEM_elneg} na systém \textit{N}+1 lineárních rovnic o \textit{N}+1 neznámých ($q_1$, $q_2$, $q_3$, ... , $q_N$, $\overline{\chi}$).
\begin{equation}
\label{EEM_matrix}
 \begin{pmatrix}
  B_{1} & \frac{\kappa}{r_{1,2}} & \cdots & \frac{\kappa}{r_{1,n}} & -1\\
  \frac{\kappa}{r_{1,2}} & B_{2} & \cdots & \frac{\kappa}{r_{2,n}} & -1 \\
  \vdots  & \vdots  & \ddots & \vdots & \vdots  \\
  \frac{\kappa}{r_{1,n}} & \frac{\kappa}{r_{2,n}} & \cdots & B_{n} & -1 \\
  1 & 1 & \cdots & 1 & 0
 \end{pmatrix} .
 \begin{pmatrix}
 q_{1} \\ q_{2} \\ \vdots \\ q_{n} \\ \overline{\chi}
 \end{pmatrix} =
 \begin{pmatrix}
 -A_{1} \\ -A_{2} \\ \vdots \\ -A_{n} \\ Q
 \end{pmatrix}
\end{equation}
Po předchozí parametrizaci (sekce \ref{param}) získáváme z matice hodnoty parciálních atomových nábojů, které jsou následně porovnány s referenčními QM náboji. Proces uzavírá statistické vyhodnocení vypočtených dat.

Na principu EEM byly později vyvinuty další empirické metody jako např. \textit{Atom-Bond Electronegativity Equalization Method} (ABEEM) nebo \textit{General Bond Electronegativity Equalization Method} (GBEEM).

\subsection{Parametrizace}
\label{param}
Parametrizace je základním nástrojem empirických modelů, jejichž cílem je reproduk\-ce experimentálních dat. Parametrizovat lze silová pole pro výpočty molekulové mechaniky nebo také \textit{ab initio} výpočty zahrnující korelační energii elektronů. Cílem parametrizace je nalezení hodnot parametrů, po jejichž integraci do empirického modelu je dosaženo co nejlepší shody experimentálních a empirických výpočtů. 

Parametrizace empirických metod pro výpočet parciálních atomových nábojů se skládá z následujících kroků:

\begin{enumerate}
\itemsep0em
    \item Výběr tréninkové sady molekul obsahující atomy, které v dostatečné míře reprezentující atomové typy, jež chceme parametrizovat
    \item Výpočet parciálních atomových nábojů tréninkové sady pomocí QM
    \item Parametrizace tréninkové sady na základě nábojů získaných v kroku 2
    \item Výpočet nábojů testovací sady molekul kvantově-mechanickou a parametrizovanou empirickou metodou
    \item Statistické vyhodnocení parametrizace
\end{enumerate}

Cílem EEM je na základě znalosti parciálních nábojů odvozených z QM určit hodnoty parametrů $A_i$, $B_i$ a $\kappa$ (viz \ref{EEM_matrix}) pro každý atom molekuly. Pro zjednodušení výpočtu jsou odpovídající řádky matice (atomy molekuly) sloučeny pod jeden atomový typ, pro který jsou počítány výše zmíněné parametry. Ty jsou na základě znalosti dílčích atomových nábojů ($q_1$, $q_2$, $q_3$, ... , $q_N$) a elektronegativity molekuly $\overline{\chi}$ získány z rovnice \ref{EEM_elneg} upravené na tvar 
\begin{equation}
   A_i + B_i\cdot q_i = \overline{\chi} - \kappa \sum_{j \neq i}^{N} \frac{q_j}{R_{ij}}
\end{equation}
Pro každou hodnotu parametru $\kappa$ ze zvoleného intervalu jsou hledány vhodné hodnoty parametrů $A_i$a $B_i$. Každému atomovému typu je pak přiřazena trojice hodnot $A_i$, $B_i$ a $\kappa$, která nejlépe reprodukuje náboje získané QM výpočty.

%- !!! pozn. parciální atomové náboje klíčovým aparátem pro výpočet silových polí v molekulových simulacích (prvni clanek)
%Gasteiger PEOE: na základě principu ekvalizace elektronegativity byly počítány např. elektronegativity atomových skupin, dipólové momenty nebo disociační energie vazeb; 
%dle PEOE vede princip úplné ekvalizace elnegativit k chemicky nepřijatelným výsledkům (článek) 

\subsection{Atomové typy}
Atomové typy jsou nástrojem parametrizace empirických metod výpočtu parciálních atomových nábojů. Definice atomových typů souvisí s charakteristickými chemickými vlastnostmi atomů (hybridizace, vazebný partner, nejvyšší řád vazby), které tyto atomové typy popisují. Každý atomový typ je definován takovou charakteristikou atomu, na základě které atom vykazuje odlišné chemické vlastnosti od jiných atomů, a je tedy vhodné pro něj definovat samostatný atomový typ. Názorným příkladem může být separace atomu uhlíku do tří atomových typů na základě jeho hybridizace. Prostorové uspořádání orbitalů uhlíku je v hybridizacích sp$^3$, sp$^2$, sp diametrálně odlišné a předurčuje tak tvorbu odlišných typů vazeb s vazebnými partnery. 

Napříč parametrizacemi empirických metod zaznamenáváme různé úrovně návrhu atomových typů, od triviálních klasifikací definujících atomové typy na základě protonových čísel až po komplexní rozdělení zahrnující nabité funkční skupiny či příslušnost k větším atomovým celkům (aromatické systémy, postranní řetězce aminokyselin). Detailní dělení atomových typů nalézáme zejména v publikacích orientujících se na parametrizaci metod pro výpočet parciálních nábojů komplexních celků, např. polypeptidů.
%- silná souvislost s parametrizací empirických metod výpočtu parciálních at. nábojů; přiřazení atomu ke zvolenému atomovému typu, pro který jsou vypočteny empirické parametry
%- otázky/úkoly bp: Jak jemná či hrubá může být kategorizace atomů, aby následná parametrizace poskytovala co nejpřesnější výsledky? + otestovat
%Jsou v rámci parametrizace některé charakteristiky atomu vhodnější pro volbu atomových typů než charakteristiky jiné? - chrakteristika = klasifikátor. Navrhnuté klasifikátory: hybridizace, nejvyšší řád vazby, atomová skupina (blízké chemické okolí)
%- různé úrovně návrhu atomových typů napříč empirickými metodami: od triviálních atomových typů klasifikujících atom podle jejich protonového čísla po komplexní rozdělení zahrnující nabité funkční skupiny či příslušnost k větším atomovým celkům (aromatická jádra, postranní řetězce aminokyselin)
%- souvislost návrhu atomových typů s typem zkoumaných molekul (pro výpočet nábojů polypeptidů definováno více atomových typů)

\section{Statistické pojmy}
\subsection{Průměrná a maximální absolutní odchylka}
Průměrná absolutní odchylka (ang. \textit{Mean Absolute Error}, MAE) je dána aritmetickým průměrem absolutních hodnot rozdílů hodnot $x_i$ a $y_i$ příslušných náhodných veličin. Po odstranění absolutních hodnot by vzorec popisoval tzv. \textit{Mean Bias Error} (MBE). 
\begin{equation}
\label{}
    MAE(X,Y) = \frac{1}{n} \sum_{i=1}^n \lvert x_i - y_i \rvert
\end{equation}

Maximální absolutní odchylka popisuje největší rozdíl nalezený mezi hodnotami $x_i$ a $y_i$ náhodných veličin X a Y.
\begin{equation}
    ABSMAX(X,Y) = max \lvert x_i - y_i \rvert
\end{equation}

\subsection{RMSD}
Veličina RMSD (z anglického \textit{Root Mean Square Deviation}, někdy uváděná též jako \textit{Root Mean Square Error}) popisuje míru odlišnosti dvojic odpovídajících hodnot ($x_i$,$y_i$) náhodných veličin \textit{X} a \textit{Y} napříč datovým souborem. Je definována jako odmocnina ze střední kvadratické chyby (\textit{Mean Square Deviation}, MSD). Stejně jako rozptyl je tato veličina kvůli kvadrátu rozdílu hodnot $x_i$ a $y_i$ citlivá na odlehlé a chybné hodnoty, které se promítají do vyšších výsledných hodnot RMSD srovnávaných datových sad. 
\begin{equation}
    RMSD(X,Y) = \sqrt{\frac{1}{n} \sum_{i=1}^n (x_i - y_i)^2} 
\end{equation}
\subsection{Pearsonův korelační koeficient}
Pro kvantifikaci funkčního vztahu dvou sledovaných veličin užíváme tzv. Pearsonův korelační koeficient (\textit{Pearson Correlation Coefficient}, PCC). PCC popisuje míru linearity závislosti veličiny \textit{Y} na veličině \textit{X} (lineární korelaci), pro popis jiných typů závislostí (např. kvadratických) není vhodný. Je definován jako
\begin{equation}
\label{pearson}
    r(X,Y) = \frac
    {\sum_{i=1}^n ((x_i - \overline{x})(y_i - \overline{y}))}
    {\sqrt{\sum_{i=1}^n (x_i - \overline{x})^2 \sum_{i=1}^n (y_i - \overline{y})^2}}
\end{equation}
kde hodnoty $x_i$, ..., $x_n$, $y_i$, ...,$y_n$ jsou \textit{i}-té prvky dvourozměrného náhodného vektoru o velikosti \textit{n} realizovaného dvěma náhodnými veličinami \textit{X} a \textit{Y}, a kde $\overline{x}$, $\overline{y}$ jsou aritmetické průměry naměřených hodnot veličin \textit{X} a \textit{Y}. 

PCC nabývá hodnot z intervalu $\langle-1, 1 \rangle$, přičemž hodnoty koeficientu blízké číslu -1 nebo 1 indikují silnou lineární korelaci mezi pozorovanými veličinami. Linearita vztahu je dobře pozorovatelná v grafu (viz obr x.x.x), kde jsou dvojice hodnot ($x_i$,$y_i$) znázorněny jako body v dvourozměrné soustavě souřadnic. Interpretace hodnoty \textit{k} Pearsonova korelačního koeficientu je následující: 
\begin{itemize}
    \item pokud je \textit{k} kladné, pak veličiny \textit{X} a \textit{Y} vykazují kladnou korelaci (pokud se hodnota veličiny \textit{Y} zvětšuje, pak hodnota\textit{ X} roste)
    \item pokud je \textit{k} záporné, pak veličiny \textit{X} a \textit{Y} vykazují zápornou korelaci (v závislosti na zvětšující se hodnotě veličiny \textit{Y} hodnota \textit{X} klesá)
    \item pokud je \textit{k} rovno 0, pak veličiny \textit{X} a \textit{Y} nejsou lineárně korelované
\end{itemize}

%- značen r, nabývá hodnot z intervalu $\langle-1, 1 \rangle$; čím více se hodnota koeficientu blíží číslu -1 nebo 1, tím je korelace mezi pozorovanými veličinami silnější
%  Statistické metody / Jiří Anděl volné na kotlářské

