\chapter{Implementace}
Tato sekce popisuje implementaci knihovny ATTYC (\textit{ATom TYpe Classification}), která přiřazuje atomům molekulového souboru atomové typy na základě zvolené vlastnosti atomu. Pro formát PDB je určen klasifikátor \verb|peptide|, klasifikátory \verb|hbo|, \verb|hybrid|, \verb|part-|\\\verb|ners| a \verb|substruct| se vztahují k formátu  SDF. Knihovna je implementována v jazyce Python ver. 3.7 a obsahuje následující soubory:

\vspace{0.4cm}
\noindent \hangpara{1.5cm}{1} \textbf{\_\_init\_\_.py} obsahuje funkci \verb|classify_atoms(input_file, classifier)|
\footnote{Argumenty \texttt{file\_output} a \texttt{screen\_output} jsou pro větší přehlednost v textu vynechány.}
spouštějící klasifikaci atomů molekulové sady za uži\-tí zvoleného klasifikátoru

\medskip
\noindent \textbf{classifier.py} definuje rozhraní pro klasifikátory implementované ve složce \textbf{\textbackslash classifiers}

\medskip
\noindent \textbf{exceptions.py} definuje výjimky (potomky třídy Exception) pro řízení běhu programu

\medskip
\noindent \hangpara{1.5cm}{1}\textbf{io.py} zpracovává všechny vstupy a výstupy (I/O) programu včetně kontroly vstupních argumentů 

\medskip
\noindent \hangpara{1.5cm}{1}\textbf{PDB\_atom\_types.txt} obsahuje názvy atomů aminokyselin definovaných nomenklaturou IUPAC a přiřazené atomové typy pro klasifikaci atomů v PDB souborech

\medskip
\noindent \hangpara{1.5cm}{1}\textbf{SMARTS\_atom\_types.txt} obsahuje SMARTS výrazy a odpovídající atomové typy pro vyhledání specifických strukturních motivů a funkčních skupin 
% (viz sekci \ref{substruct})

\medskip
\noindent\textbf{\textbackslash classifiers} obsahuje klasifikátory, které rozdělují atomy do atomových typů na základě:

\medskip
\textbf{\_\_init\_\_.py}

\vspace{0.01cm}
\textbf{hbo.py} nejvyššího řádu vazby (angl. \textit{highest bond orded})

\vspace{0.01cm}
\textbf{hybrid.py} hybridizace

\vspace{0.01cm}
\textbf{partners.py}\ vazebných partnerů

\vspace{0.01cm}
\textbf{peptide.py} pozice atomu v rámci aminokyseliny
    
\vspace{0.01cm}
\textbf{substruct.py} příslušnosti ke strukturnímu motivu nebo funkční skupině

\bigskip
Výstup knihovny ATTYC je specifikován argumenty \verb|file_output| a \verb|screen_output| funkce \verb|classify_atoms(...)|. Pokud má argument \verb|file_output| hodnotu \verb|True|, jsou přiřazené atomové typy molekul ze vstupního souboru zapsány do textového souboru ve složce ATTYC\_outputs. V případě použití argumentu \verb|screen_output| je na standardní výstup vypsána statistika přiřazených atomových typů.


%%%%%%%%%%%%%%%% nova verze, zacinam s fci classify_atoms
\section{Spuštění klasifikace}
% \texttt{classify\_atoms(input\_file, classifier)}
Klasifikace atomů do atomových typů ve vstupním souboru je spuštěna voláním fun\-kce \verb|classify_atoms(input_file, classifier, file_output, screen_output)|, která je 
implementována v modulu \textbf{\_\_init\_\_.py}. Funkce je z modulu \textbf{main.py} volána příkazem \verb|attyc.classify_atoms(input_file, classifier)|. Tento příkaz spouští klasifikaci atomů z libovolného programu v jazyce Python, do kterého byla integrována knihovna ATTYC. Pro spuštění klasifikace je nutné do interpretu Pythonu programu nainstalovat knihovnu RDKit. 

V rámci funkce \verb|classify_atoms(input_file, classifier)| je kontrolována e\-xistence vstupního molekulového souboru a zvoleného klasifikátoru. Pokud je klasifikátor (tzn. podtřída třídy Classifier) implementován ve složce \textbf{\textbackslash classifiers} a je povolen pro daný molekulový formát, je vytvořena jeho instance. Tato instance přiřazuje atomům v molekulové sadě odpovídající atomový typ.

%voláním metody \verb|get_atom_types(molecule)|, která je volána skrze metodu \verb|classify_atoms_by_classifier(moleculeset, is_pdb)| třídy Classifier. Metoda \verb|get_atom_types(molecule)| je abstraktní metodou třídy Classifer a je implementována všemi podtřídami třídy Classifier.

\section{Dědičnost třídy Classifier}
Třída Classifier definuje rozhraní pro podtřídy definované v modulech ve složce \textbf{\textbackslash classifiers}. Všechny podtřídy třídy Classifier obsahují atribut \verb|assigned_atom_types|, do~kterého ukládají přiřazené atomové typy, a atribut \verb|name|. Kromě getterů uvedených atribu\-tů dědí podtřídy metodu \verb|classify_atoms_by_classifier(moleculeset, is_pdb)|. Ta\-to metoda volá abstraktní metodu \verb|get_atom_types(molecule)|, kterou každá podtřída třídy Classifier implementuje. Podtřídy, které představují klasifikátory \verb|substruct| a \verb|peptide|, obsahují kromě výše uvedených atributů také atributy \verb|SMARTS_atom_types| a~\verb|PDB_atom_types|, do kterých ukládají data z externích souborů potřebná pro klasifikaci atomů.

Každý modul ve složce \textbf{\textbackslash classifiers} obsahuje právě jednu podtřídu třídy Classifier, tedy právě jeden klasifikátor. Moduly v této složce spolu nijak neinteragují, knihovna je tak snadno rozšiřitelná o další libovolné klasifikátory. 

\bigskip
\begin{figure}[h]
    \centering
    \includegraphics[width=7cm]{example-image-a}
    \caption{UML diagram tříd popisující třídu Classifier a její podtřídy.}
    \label{classes_UML}
\end{figure}


\section{Přiřazení atomových typů}
% \texttt{get\_atom\_types(molecule)}
Každá podtřída třídy Classifier (dále jen 'klasifikátor') implementuje abstraktní metodu nadtřídy \verb|get_atom_types(molecule)|, která atomům molekuly přiřazuje odpovídající atomové typy. Klasifikátory \verb|hbo|, \verb|hybrid| a \verb|partners| implementují tuto metodu poměrně triviálně, neboť využívají příslušné metody tříd Atom a Bond z knihovny RDKit. Názvy metod použitých pro klasifikaci jsou uvedeny v tabulce níže.

Pro klasifikátory \verb|substruct| a \verb|peptide| nejsou v třídě Atom v knihovně RDKit implementovány triviální metody, atomy jsou proto těmito klasifikátory klasifikovány pomocí dat z externích souborů. Struktura vytvořených externích souborů a logika těchto klasifikací je popsána v následujících sekcích. 

\begin{table}[h]
\begin{center}
\label{atom_rdkit_methods}
\renewcommand{\arraystretch}{1.3}
    \begin{small}
    \hspace{7mm}\begin{tabular}{c|l}
        % \hline
        \verb|hbo| & Atom.GetBonds(), Bond.GetBondTypeAsDouble() \\
        \hline
        \verb|hybrid| & Atom.GetHybridization() \\
        \hline
        \verb|partners| & Atom.GetNeighbors() \\
        % \hline
    \end{tabular}
    \end{small}
    \caption{Klíčové metody z knihovny RDKit použité pro přiřazení atomových typů klasifikátory \texttt{hbo}, \texttt{hybrid} a \texttt{substruct}.}
\end{center}
\end{table}
\subsection{Klasifikátor \texttt{substruct}}
% \label{substruct}
Klasifikátor \verb|substruct| rozděluje atomy do atomových typů na základě příslušnosti k~charakteristickým strukturním celkům. Byl implementován s cílem reprodukovat atomové typy, které byly úspěšně použity pro parametrizaci empirických metod výpočtu parciálních atomových nábojů \cite{attyp1, attyp2}. 

Strukturní motivy a funkční skupiny jsou v molekulách detekovány pomocí výrazů SMARTS, a to metodou třídy Mol \verb|GetSubstructMatches(SMARTS_pattern)| implementovanou v knihovně RDKit. 
Příkaz \verb|molecule.GetSubstructMatches(SMARTS_pa|\\\verb|ttern)|, kde \verb|molecule| je instance třídy Mol, 
vrací n-tici (datový typ \textit{tuple}) n-tic, které  obsahují prvky typu integer. Tato čísla označují atomy molekuly vyhovující danému SMARTS dotazu. Čísla atomů vycházejí z pořadí, ve kterém jsou atomy definovány v SDF souboru. 
%a odpovídají pořadí definice atomů v SDF souboru.
Příklad lze vidět na obr. \ref{smarts_tuples}.

\lstset{language=Python, keywordstyle=\color{blue}, basicstyle=\small\ttfamily, commentstyle=\color{orange}, label={smarts_exm}}

% pridat do obrazku vice SMARTSu pro nazornost?
\begin{figure}[h]
\label{smarts_tuples}
\begin{lstlisting}
# molecule je instance tridy Mol
SMARTS_pattern = "[SX4](=[OX1])(=[OX1])[OX2,OX1-]"
pattern_atoms = molecule.GetSubstructMatches(SMARTS_pattern)
for atom_tuple in pattern_atoms:
    print(atom_tuple, get_elements(atom_tuple))
print(pattern_atoms)

Python 3.7.0 (default, Apr 9 2019, 10:31:47)
>>> (5, 7, 8, 11) ('S', 'O', 'O', 'O')
>>> (21, 22, 26, 29) ('S', 'O', 'O', 'O')
>>> ((5, 7, 8, 11), (21, 22, 26, 29))
\end{lstlisting}
\cprotect\caption{Ukázka funkce \verb|GetSubstructMatches(SMARTS_pattern)|. Pomo\-cí SMARTS výrazu
jsou v molekule vyhledány sulfonové skupiny S(=O)$_2$OH nebo její anionty. Indexy atomů jsou v n-ticích zapsány v pořadí odpovídajícímu prvkům ve SMARTS výrazu. Reálná implementace metody pro získání uvedeného výstupu, \verb|molecule.GetSubstructMatches(Chem.MolFromSmarts(SMARTS_pattern))|, je v textu i v obrázku pro názornost syntakticky zjednodušena.}
\end{figure}

Klíčovým prvkem klasifikace je 
- na základě znalosti výstupů funkce \verb|GetSubstructMatches(SMARTS_pattern)| pro jsou v souboru SMARTS\_atom\_types.txt definovány atomové typy 


- validace volně dostupných SMARTS výrazů užitím online nástroje LiteMol a jejich úprava -> zpřesnění dotazů, odstranění redundantních výstupů 
- strukturní motivy hledány v pořadí, v jakém jsou definovány SMARTS vstupy v souboru SMARTS\_atom\_types.txt - pořadí SMARTS vstupů ovlivňuje výsledný atomový typ, který je atomu přiřazen
- SMARTS vstupy definovány od nejzvláštnjších k nejvíce obecným 
- atomům, které nebyly vyhledáváním strukturních motivů označeny, je přiřazen atomový typ 'plain'


\subsection{Klasifikátor \texttt{peptide}}
%- zpracování PDB souborů, mmCIF formát není knihovnou RDKit podporován
%- PDB de facto zastaralý, jako standard jej nahradil formát mmCIF
%- pro účely bakalářské práce není omezením; cíl: otestovat, zda je navržená klasifikace použitelná (pozdější integrace openbabelu pro konverzi souborů?)



 



