\chapter{Implementace}
Tato sekce popisuje implementaci knihovny ATTYC (\textit{ATom TYpe Classification}), která přiřazuje atomům molekulového souboru atomové typy na základě zvolené vlastnosti atomu. Pro formát PDB je určen klasifikátor \verb|peptide|, klasifikátory \verb|hbo|, \verb|hybrid|, \verb|part-|\\\verb|ners| a \verb|substruct| se vztahují k formátu  SDF. Knihovna je implementována v jazyce Python ver. 3.7 a obsahuje následující soubory:

\vspace{0.4cm}
\noindent \hangpara{1.5cm}{1} \textbf{\_\_init\_\_.py} obsahuje funkci \verb|classify_atoms(input_file, classifier)|\footnote{Argumenty \texttt{file\_output} a \texttt{screen\_output} jsou pro větší přehlednost v textu vynechány.}
spouštějící klasifikaci atomů molekulové sady za uži\-tí zvoleného klasifikátoru

\medskip
\noindent \textbf{classifier.py} definuje rozhraní pro klasifikátory implementované ve složce \textbf{\textbackslash classifiers}

\medskip
\noindent \textbf{exceptions.py} definuje výjimky (potomky třídy Exception) pro řízení běhu programu

\medskip
\noindent \hangpara{1.5cm}{1}\textbf{io.py} zpracovává všechny vstupy a výstupy (I/O) programu včetně kontroly vstupních argumentů 

\medskip
\noindent \hangpara{1.5cm}{1}\textbf{PDB\_atom\_types.txt} obsahuje názvy atomů aminokyselin definovaných nomenklaturou IUPAC a přiřazené atomové typy pro klasifikaci atomů v PDB souborech

\medskip
\noindent \hangpara{1.5cm}{1}\textbf{SMARTS\_atom\_types.txt} obsahuje SMARTS výrazy a odpovídající atomové typy pro vyhledání specifických strukturních motivů a funkčních skupin 
% (viz sekci \ref{substruct})

\medskip
\noindent\textbf{\textbackslash classifiers} obsahuje klasifikátory, které rozdělují atomy do atomových typů na základě:

\medskip
\textbf{\_\_init\_\_.py}

\vspace{0.01cm}
\textbf{hbo.py} nejvyššího řádu vazby (angl. \textit{highest bond orded})

\vspace{0.01cm}
\textbf{hybrid.py} hybridizace

\vspace{0.01cm}
\textbf{partners.py}\ vazebných partnerů

\vspace{0.01cm}
\textbf{peptide.py} pozice atomu v rámci aminokyseliny
    
\vspace{0.01cm}
\textbf{substruct.py} příslušnosti ke strukturnímu motivu nebo funkční skupině

\bigskip
Výstup knihovny ATTYC je specifikován argumenty \verb|file_output| a \verb|screen_output| funkce \verb|classify_atoms(...)|. Pokud má argument \verb|file_output| hodnotu \verb|True|, jsou přiřazené atomové typy molekul ze vstupního souboru zapsány do textového souboru ve složce ATTYC\_outputs. V případě použití argumentu \verb|screen_output| je na standardní výstup vypsána statistika přiřazených atomových typů.


%%%%%%%%%%%%%%%% nova verze, zacinam s fci classify_atoms
\section{Spuštění klasifikace}
% \texttt{classify\_atoms(input\_file, classifier)}
Klasifikace atomů do atomových typů vstupního molekulového souboru je spuštěna voláním fun\-kce \verb|classify_atoms(input_file, classifier)|, která je 
implementována v modulu \textbf{\_\_init\_\_.py}. Funkce je z modulu \textbf{main.py} volána příkazem \verb|attyc.class|\\ \verb|ify_atoms(input_file, classifier)|. Tento příkaz spouští klasifikaci atomů z libovolného programu v jazyce Python, do kterého byla integrována knihovna ATTYC. Pro spuštění klasifikace je nutné do interpretu Pythonu daného programu nainstalovat knihovnu RDKit. 

V rámci funkce \verb|classify_atoms(input_file, classifier)| je kontrolována e\-xistence vstupního molekulového souboru a zvoleného klasifikátoru. Pokud je klasifikátor (tzn. podtřída třídy Classifier) implementován ve složce \textbf{\textbackslash classifiers} a je povolen pro daný molekulový formát, je vytvořena jeho instance. Tato instance přiřazuje atomům v molekulové sadě odpovídající atomový typ.

%voláním metody \verb|get_atom_types(molecule)|, která je volána skrze metodu \verb|classify_atoms_by_classifier(moleculeset, is_pdb)| třídy Classifier. Metoda \verb|get_atom_types(molecule)| je abstraktní metodou třídy Classifer a je implementována všemi podtřídami třídy Classifier.

\section{Dědičnost třídy Classifier}
Třída Classifier definuje rozhraní pro podtřídy definované v modulech ve složce \textbf{\textbackslash classifiers}. Všechny podtřídy třídy Classifier obsahují atribut \verb|assigned_atom_types|, do~kterého ukládají přiřazené atomové typy, a atribut \verb|name|. Kromě getterů uvedených atribu\-tů dědí podtřídy metodu \verb|classify_atoms_by_classifier(moleculeset, is_pdb)|. Ta\-to metoda volá abstraktní metodu \verb|get_atom_types(molecule)|, kterou každá podtřída třídy Classifier implementuje. Podtřídy, které představují klasifikátory \verb|substruct| a \verb|peptide|, obsahují kromě výše uvedených atributů také atributy, do kterých ukládají data z externích souborů potřebná pro klasifikaci atomů.

Každý modul ve složce \textbf{\textbackslash classifiers} obsahuje právě jednu podtřídu třídy Classifier, tedy právě jeden klasifikátor. Moduly v této složce spolu nijak neinteragují, knihovna je tak snadno rozšiřitelná o další libovolné klasifikátory. 

\bigskip
\begin{figure}[h]
    \centering
    \includegraphics[width=7cm]{example-image-a}
    \caption{UML diagram tříd popisující třídu Classifier a její podtřídy.}
    \label{classes_UML}
\end{figure}

    
\section{Přiřazení atomových typů}
% \texttt{get\_atom\_types(molecule)}
Každá podtřída třídy Classifier (dále jen 'klasifikátor') implementuje abstraktní metodu nadtřídy \verb|get_atom_types(molecule)|, která atomům molekuly přiřazuje odpovídající atomové typy. Klasifikátory \verb|hbo|, \verb|hybrid| a \verb|partners| implementují tuto metodu triviálně, neboť využívají příslušné metody tříd Atom a Bond z knihovny RDKit (tab.  \ref{atom_rdkit_methods}). 
%Názvy metod použitých pro klasifikaci jsou uvedeny v tabulce níže.

Pro klasifikátory \verb|substruct| a \verb|peptide| nejsou v třídě Atom v knihovně RDKit implementovány triviální metody, atomy jsou proto těmito klasifikátory klasifikovány pomocí dat z externích souborů. Struktura vytvořených externích souborů a logika  klasifikací je popsána v následujících sekcích. 

\begin{table}[h]
\begin{center}
\label{atom_rdkit_methods}
\renewcommand{\arraystretch}{1.3}
    \begin{small}
    \hspace{7mm}\begin{tabular}{c|l}
        % \hline
        \verb|hbo| & Atom.GetBonds(), Bond.GetBondTypeAsDouble() \\
        \hline
        \verb|hybrid| & Atom.GetHybridization() \\
        \hline
        \verb|partners| & Atom.GetNeighbors() \\
        % \hline
    \end{tabular}
    \end{small}
    \caption{Klíčové metody z knihovny RDKit použité pro přiřazení atomových typů klasifikátory \texttt{hbo}, \texttt{hybrid} a \texttt{partners}.}
\end{center}
\end{table}
\subsection{Klasifikátor 'substruct'}
% \texttt{substruct}
% \label{substruct}
Klasifikátor \verb|substruct| rozděluje atomy do atomových typů na základě příslušnosti k~charakteristickým strukturním celkům. Byl implementován s cílem reprodukovat atomové typy, které byly úspěšně použity pro parametrizaci empirických metod výpočtu parciálních atomových nábojů \cite{attyp1, attyp2}. 

Strukturní motivy jsou v molekulách detekovány pomocí výrazů SMARTS (kap. \ref{SMARTS}) metodou  \verb|GetSubstructMatches(SMARTS_pattern)|\footnote{Metoda  \texttt{GetSubstructMatches(Chem.MolFromSmarts(SMARTS\_pattern))} je v textu pro názornost syntakticky zjednodušena.} třídy Mol z kni\-hovny RDKit. 
Příkaz \verb|molecule.GetSubstructMatches(SMARTS_pattern)|, kde \verb|molecule| je instance třídy Mol, 
vrací n-tici (datový typ \textit{tuple}) n-tic, které  obsahují prvky typu integer. Tato čísla označují atomy molekuly vyhovující danému SMARTS dotazu. 
%Čísla atomů vycházejí z pořadí, ve kterém jsou atomy definovány v SDF souboru.
Kód níže ukazuje užití SMARTS výrazu k vyhledání sulfonové skupiny S(=O)$_2$OH nebo jejích aniontů.


\lstset{language=Python, keywordstyle=\color{blue}, basicstyle=\small\ttfamily, commentstyle=\color{orange}, label={smarts_exm}}
% pridat do obrazku vice SMARTSu pro nazornost?
\begin{lstlisting}
SMARTS_pattern = "[SX4](=[OX1])(=[OX1])[OX2,OX1-]"
pattern_atoms = molecule.GetSubstructMatches(SMARTS_pattern)
for atom_tuple in pattern_atoms:
    print(atom_tuple, get_elements(atom_tuple))
print(pattern_atoms)

Python 3.7.0 (default, Apr 9 2019, 10:31:47)
(5, 7, 8, 11) ('S', 'O', 'O', 'O')
(21, 22, 26, 29) ('S', 'O', 'O', 'O')
((5, 7, 8, 11), (21, 22, 26, 29))
\end{lstlisting}

Pro klasifikaci atomů je klíčovým souborem \textbf{SMARTS\_atom\_types.txt}. Soubor obsahuje SMARTS výrazy, pomocí nichž jsou v molekule hledány strukturní motivy, a atomové typy, které jsou atomům strukturního motivu explicitně přiřazeny.
Pořadí detekce strukturních motivů odpovídá pořadí SMARTS vstupů v souboru. Některé atomy molekuly mohou příslušet více strukturním motivům, které SMARTS vstupy detekují. Uspořádání SMARTS vstupů v souboru tak ovlivňuje výsledný atomový typ, který je atomu přiřazen. Atomové typy přiřazené jednotlivým atomům lze v souboru předefinovat, stejně jako změnit pořadí SMARTS výrazů, a upravit tak logiku klasifikace a optimalizovat výsledky parametrizace.
\begin{figure}[h]
    \centering
    \includegraphics[width=7cm]{example-image-a}
    \caption{Struktura souboru \textbf{SMARTS\_atom\_types.txt}. Pro každý SMARTS výraz jsou definovány atomové typy atomů dané funkční skupiny.}
    \label{SMARTS_file}
\end{figure}

SMARTS výrazy v souboru \textbf{SMARTS\_atom\_types.txt} byly převzaty ze stránek společnosti Daylight Chemical Information System, Inc. \cite{SMARTS_exm}, pro účely klasifikace atomů však musely být ve většině případů dodatečně upraveny. Vybrané SMARTS výrazy byly rozšířeny o detekci většího počtu atomů nebo byly upraveny tak, aby %se zamezilo redundanci výsledků jednotlivých SMARTS dotazů.
se výsledky jednotlivých SMARTS dotazů nepřekrývaly. 

Úpravu pro snížení redundance výsledků ilustruje následující příklad. Vyhledání karboxylové skupiny za užití výrazu \verb|[CX3](=O)[OX2][H]| je v externím souboru následováno detekcí alkoholové skupiny pomocí výrazu \verb|[CX4,c][OX2][H]|. Pokud by nebyla specifikována vaznost uhlíku ('\verb|CX4|'), vyhledal by daný SMARTS výraz i OH skupiny, které by byly součástí dříve určených karboxylových skupin. Rozšíření SMARTS dotazů o detekci více atomů se ve většině případů týkalo detekce vodíků. Pro vyhledání primárních aminů je v uvedeném online zdroji specifikován výraz \verb|[NX3;H2!$(NC=O)]|. Tento výraz byl upraven
na dotaz \verb|[NX3;!\$(NC=O)]([H])[H]|, kterým jsou narozdíl od původního výrazu vodíky již detekovány.
 
\subsection{Klasifikátor 'peptide'}
% \texttt{peptide}
Klasikace atomů peptidových řetězců přiřazuje atomové typy obdobně jako výše uvedený klasifikátor \verb|substruct|, tedy na základě příslušnosti atomů ke strukturním celkům. Logika vstupního souboru klasifikátoru \verb|peptide| se od struktury externího souboru klasifikátoru \verb|substruct| liší, neboť nepracuje s vyhledáváním strukturních motivů pomocí SMARTS výrazů. 

Symboly atomových typů aminokyselin popsané nomenklaturou IUPAC (CA, CB, HXT, OXT,...) popisují atomy všech proteinogenních residuí. Jednotlivé atomové typy však nijak nereflektují vlastnosti atomů, které popisují, chemická okolí ato\-mů označených stejným atomovým typem se tak často liší (srov. obr. \ref{aa_different_atom_types}). Pro implementaci klasifikátoru \verb|peptide| tak bylo klíčové provést rešerši existujících atomových typů aminokyselin, určit napříč všemi aminokyselinami výskyt těchto atomových typů a každé dvojici 'atomový typ  nomenklatury IUPAC - aminokyselina' přiřadit atomový typ reflektující chemické okolí daného atomu. Navržená klasifikace atomů aminokyselin byla implementována s cílem reprodukovat atomové typy, které byly použity pro úspěšnou parametrizaci empirických metod \cite{GDAC, attyp_peptides}.

\begin{figure}[h]
\label{aa_different_atom_types}
\begin{center}
\includegraphics[width=7cm]{example-image-a}
\caption{Strukturní vzorce histidinu a kyseliny asparagové s atomovými typy nomenklatury IUPAC. Atomy CG a HD2 jsou v molekulách součástí odlišných strukturních celků, je tedy vhodné každému z nich přiřadit jiný atomový typ.}
\end{center}
\end{figure}

% vložit ilustrační obrázek souboru PDB_atom_types.txt? 


 



